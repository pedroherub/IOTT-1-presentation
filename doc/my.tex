\documentclass[
    % draft,                             % 草稿模式
    aspectratio=169,                   % 使用 16:9 比例
]{beamer}
\mode<presentation>
\usetheme[
    % navigation=subsections,            % 使用子章节进度显示
    % lang=en,                           % 使用英文
    % cjk=true,                          % 使用CJK而不是ctex
    color=red,                         % 使用红色主题
    % pattern=all,                        % 使用全图案装饰
    % gbt=bibtex,                        % 使用 gbt (使用 bibtex 编译)
]{sjtubeamermin}
\usecolortheme[]{beaver}                 % 使用其他颜色主题
\addbibresource{ref.bib}               % gbt!=bibtex

\begin{document}
    % \institute[School of Electrical Engineering]{数学科学学院}   % 组织
    \institute[Department of Automation]{自动化系}   % 组织
    \logo{
        \includegraphics{cnlogored.pdf}  % 重定义 logo
    }
    \titlegraphic{                         % 标题图像
        \begin{stampbox}
            \includegraphics[width=0.3\textwidth]{head.png}
        \end{stampbox}
    }
    \title{Assignment 1: Coffeemaker control system}  % 标题
    \subtitle{IoT application system development}         % 副标题
    \author{Pedro Hernández Rubio (021032990002)}                  % 作者
    \date{\today}                          % 日期  
    \maketitle                             % 创建标题页

\part{第一部分 IoT application system development}

% 使用节目录
\AtBeginSection[]{
    \begin{frame}
        % \tableofcontents[currentsection]           % 传统节目录             
        \sectionpage                   % 节页
    \end{frame}
}

% 使用小节目录
% \AtBeginSubsection[]{                  % 在每小节开始
%     \begin{frame}
%         \tableofcontents[currentsection,currentsubsection]             % 传统小节目录             
%         \subsectionpage                % 小节页
%     \end{frame}
% }

\section{第 1 节 The IoT idea}

\subsection{第 1 小节 Problem}

    \begin{frame}
        \frametitle{Problem}

        \paragraph{Goal} Research the use of \alert{Blockchain technologies} and \alert{P2P storage systems} for IoT systems management.

        \begin{itemize}
            \item \alert{Decentralized}: IoT devices data are not required to be sent to centralized services (default trust): data can be stored securely on the peers in the network, anb blockchain assure its authenticity and control authorized accesses.
            \item \alert{Private-by-design IoT}: providing privacy, robust systems and no single point of failure.
        \end{itemize}

        \begin{block}{Security}
            All operations involving IoT devices in the network can be registered and authenticated in the blockchain: data creation, data modification or data deletion. IoT devices security could be drammatically enhanced
        \end{block}

    \end{frame}

\subsection{第 2 小节 Solution}

    \begin{frame}
        \frametitle{Solution}

        \begin{block}{Beyond criptocurrencies}
            Paradigm of decentralizing services that so far have depend exclusively of centralized trusting authorities. But possible applications are endless.
        \end{block}

        \paragraph{Research} Several use cases were found on the literature:

        \begin{itemize}
            \item \alert{Data store management}: data collected by IoT devices can be stored in a blockchain as a single point of truth.\cite{sia}
            \item \alert{Authentication service}: by public key cryptography, all IoT devices can be securely identified in order to communicate safely among them.\cite{Axon2015PrivacyawarenessIB}
            \item \alert{Exchanging data}: data collected by devices on one specific network can be traded (or exchanged) according to common establised policies.
        \end{itemize}

    \end{frame}

\section{第 2 节 Components}

\subsection{第 1 小节 Hardware}

    \begin{frame}
        \frametitle{NodeMCU}

        \begin{block}{Integrity}
            IoT applications can be securely developed on top of stable blockchain networks, by building a layered architecture with thin clients.\cite{Ali2016BlockstackDA} Thus, consensus protocols and large enough nodes community make possible attacks mostly unfeasible
        \end{block}
        \begin{block}{Anonimity}
            As most of blockchain networs are public, full anonimity is not achieved (IoT devices can still be linked to their respective owners) -> \alert{pseudonimity}
            Methods for not associating public keys to IoT devices' IDs are being investigated.\cite{Axon2015PrivacyawarenessIB}
        \end{block}

    \end{frame}

    \begin{frame}
        \frametitle{Temperature sensor DS18B20}

        \begin{block}{Adaptability}
            Poor scalability and low throughput of transactions are the biggest issues related to IoT systems. An acceptable trade-off between security and scalability must be determined.\cite{conoscenti}
        \end{block}

        Mitigations:
        \begin{itemize}
            \item \alert{Layered architecture}: IoT applications built on top of scalable blockchain network.
            \item \alert{Data storage}: allowing thin clients to delegate some data in the blockchain, as many IoT devices have limited resources.\cite{Axon2015PrivacyawarenessIB}
        \end{itemize}

    \end{frame}

\subsection{第 2 小节 Software}

    \begin{frame}
        \frametitle{Arduino IDE}

        \begin{block}{Integrity}
            IoT applications can be securely developed on top of stable blockchain networks, by building a layered architecture with thin clients.\cite{Ali2016BlockstackDA} Thus, consensus protocols and large enough nodes community make possible attacks mostly unfeasible
        \end{block}
        \begin{block}{Anonimity}
            As most of blockchain networs are public, full anonimity is not achieved (IoT devices can still be linked to their respective owners) -> \alert{pseudonimity}
            Methods for not associating public keys to IoT devices' IDs are being investigated.\cite{Axon2015PrivacyawarenessIB}
        \end{block}

    \end{frame}

    \begin{frame}
        \frametitle{Libraries}

        \begin{block}{Adaptability}
            Poor scalability and low throughput of transactions are the biggest issues related to IoT systems. An acceptable trade-off between security and scalability must be determined.\cite{conoscenti}
        \end{block}

        Mitigations:
        \begin{itemize}
            \item \alert{Layered architecture}: IoT applications built on top of scalable blockchain network.
            \item \alert{Data storage}: allowing thin clients to delegate some data in the blockchain, as many IoT devices have limited resources.\cite{Axon2015PrivacyawarenessIB}
        \end{itemize}

    \end{frame}

\section{第 3 节 Development}

\subsection{第 1 小节 Phase 1: Arduino IDE \& sensor}

    \begin{frame}
        \frametitle{System design}
        
        \paragraph{Goal} Research decentralized access control solutions for IoT systems management.\cite{ouaddah}

        \begin{itemize}
            \item \alert{DOAuth}: Decentralized Open Authentication (heavyweight for IoT)
            \item \alert{FairAccess}: for non-constrained resources (without memory problems)
            \item \alert{IBM Adept}: Autonomous Decentralized P2P Telemetry
        \end{itemize}

        \begin{block}{Scalability problem}
            Access management to billions of constrained IoT devices is quite a challenge. An easy-to-manage, generic and scalable solution is the goal.
        \end{block}
    \end{frame}

\subsection{第 2 小节 Phase 2: IoT platform (ThingSpeak)}

    \begin{frame}
        \frametitle{System design}
        
        \begin{block}{Scalable Access Management}
            Because of scalability issues, a feasible solution proposes that access control policies are stored in the blockchain layer (smart contract). Threrefore, separating IoT devices with the blockchain\cite{novo}
        \end{block}

        \paragraph{Decision} Main advantages of this access control in IoT:

        \begin{itemize}
            \item \alert{Mobility}: encapsulating administrative domains
            \item \alert{Accessibility}: policies always available
            \item \alert{Concurrency}: many managers for each device
            \item \alert{Lightweight}: devices straightforward plugin
            \item \alert{Scalability}: directly connected to blockchain
            \item \alert{Transparency}: locations and policies hidden
        \end{itemize}
    \end{frame}

\subsection{第 3 小节 Phase 3: Node-RED \& Telegram}

    \begin{frame}
        \frametitle{System design}

            \begin{figure}
                \centering
                \begin{stampbox}
                    \includegraphics[height=0.5\textheight]{architecture.png}
                \end{stampbox}
                \caption{Decentralized Access Control Architecture\cite{novo}}
            \end{figure}

        % \begin{block}{Smart contract}
        %     Access control management information is stored in the blockchain layer.\cite{novo}
        % \end{block}
        
        % \begin{figure}
        %     \centering
        %     \begin{stampbox}
        %         \includegraphics[height=0.4\textheight]{architecture.png}
        %     \end{stampbox}
        %     \caption{Decentralized Access Control Architecture}
        % \end{figure}
    \end{frame}

    \begin{frame}
        \frametitle{System design}

        \paragraph{Architecture} Components classification:

        \begin{block}{Outside blockchain network}
            \begin{itemize}
                \item \alert{Wireless sensor networks}: IoT devices domain
                \item \alert{Management hubs}: intermediary between devices and blockchain
            \end{itemize}
        \end{block}

        \begin{block}{Inside blockchain network}
            \begin{itemize}
            \item \alert{Managers}: smart contract managing
            \item \alert{Agent node}: smart contract access
            \item \alert{Smart contract}: access control system rules
            \item \alert{Blockchain network}: private
            \end{itemize}
        \end{block}
    \end{frame}

\subsection{第 4 小节 Phase 4: }

    \begin{frame}[fragile]          % 注意添加 fragile 标记
        \frametitle{Phase 4}
        % 代码块参数:语言,标题
        % 请减少代码初始的缩进
        \begin{codeblock}[language=c++]{C++代码}
#include<iostream>

int main(){
    // Console Output
    std::cout << "Hello, SJTU!" << std::endl;
    return 0;
}
        \end{codeblock}
    \end{frame}

    \begin{frame}
        \frametitle{图}
        \begin{figure}
            \centering
            \begin{stampbox}
                \includegraphics[height=0.3\textheight]{plant.jpg}
            \end{stampbox}
            \caption{图片标题\cite{viman}}
        \end{figure}
    \end{frame}

    \begin{frame}
        \frametitle{Issues}
        \begin{multicols}{2}
        \begin{table}
            \caption{表格标题\cite{pgfplotstableman}}
            \pgfplotstabletypeset[
                columns/Quick/.style={dec sep align},
                columns/Cocktail/.style={dec sep align},
                column type=r,
                % fixed zerofill,
            ]{test.csv}
        \end{table}
        
        \begin{figure}
            \input{testgraph.tex}
            \caption{统计图标题\cite{pgfplotsman}}
        \end{figure}
        \end{multicols}
    \end{frame}


% gbt=bibtex
\part{References 参考文献}
    \begin{frame}[allowframebreaks]
        \printbibliography[title=References 参考文献]    % gbt!=bibtex
        % \bibliography{ref.bib}             % gbt=bibtex
    \end{frame}

    \makebottom     % 创建尾页  % 非标准命令

\end{document}